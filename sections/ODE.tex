\part{Ordinary Differential Equations}
\setcounter{section}{0}

% Definition von ODE
\Def A \textit{differential equation} is an equation where the unknown is a function f, and the equation relates $f(x)$ with values of derivatives $f^{(i)}$ \textbf{at the same point} $x$.

\Def Ordinary $\iff$ One (input) variable only

\sep

\Def A Linear ODE is an equation of the form
\[ y^{(k)} + a_{k - 1} y^{(k - 1)} + \cdots + a_0 y = D(y) = b \] 
where $y = f(x)$ is the unknown function \\
$a_{k - 1}(x), ..., a_0(x), b(x)$ are continuous functions. \\

\Def Homogenous $ \iff b(x) = 0$ 

\Def[Initial Condition] A set of equations
\[y(x_0) = y_0, y'(x_0) = y_1, ..., y^{(k - 1)}(x_0) = y_{k - 1} \]

\sep

\Theorem[2.2.3] $I \subset \R$,  linear ODE of order $k \geq 1$
\begin{enumerate}
\item[(1)] Let $S_0$ be the set of solutions for $b = 0$. Then is $S_0$ a vector space of dimension k.
\item[(2)] For any initial conditions, there is a unique solution $f \in S_0$, s.t.
\[y(x_0) = y_0, y'(x_0) = y_1, ..., y^{(k - 1)}(x_0) = y_{k - 1} \]

\item[(3)] For an arbitrary b, the set of solutions is $S_b = \{f + f_p | f \in S_0\}$, where $f_p$ is a particular solution
\item[(4)] For any initial value problem, there is a unique solution $f \in S_b$
\end{enumerate}
\Bem If $b \neq 0$, then $S_b$ is not a vector space \\
\Bem If $f_1, f_2$ are solutions for $b_1(x), b_2(x)$, \\ $f_1 + f_2$ is a solution for $b_1(x) + b_2(x)$




\section{Linear ODEs of order 1}
\Procedure Consider $y' + ay = b $
\begin{enumerate}
\item[1.] Solve homogeneous equation $y' + ay = 0$ 
\[ f_0(x) = z \cdot e^{-A(x)} \text{ for } z \in \C\]
\item[2.] Find a solution of the inhomogeneous equation $f_p$, then $S_b  = f_p + S_0$.
\begin{itemize}
  \item Guess: $b(x)$ should resemble $f_p$
  \item Variation of Constants (Assume constants of $S_0$ are functions)
  \item Formula: $f_p(x) = \int b(x) \cdot e^{A(x)} dx \cdot e^{-A(x)}$
\end{itemize}
\end{enumerate}
\Bem The solutions are given by $f_0 + z f_1$, where $z \in \C$ and $f_1$ is a basis of $S$ \\
\Bem To solve the real value problem $f(x_0) = y_0$, one can solve $f_0(x_0) + z f_1(x_0) = y_0$ \\
\Bem If $a \in \R$, then there exists $f_0, f_1 \in \R$ \\

\section{Lin. ODE with constant coefs. }
The equation takes the form: Let $a_{k - 1}, ..., a_0 \in \C$
\[ y^{(k)} + a_{k - 1} y^{(k - 1)} + ... + a_0 y = b(x) \] 
\sep

\Intuition We look for solutions of the form $f(x) = e^{\lambda x}, \lambda \in \C$
\begin{align*}
0 &=y^{(k)} + a_{k - 1} y^{(k - 1)} + ... + a_0 y \\
&=  e^{\lambda x}( \lambda^k + a_{k - 1} \lambda^{k - 1} + \cdots + a_1 \lambda + a_0) \\
&= e^{\lambda x} P(\lambda)
\end{align*}

$\implies f$ is a solution if and only if $P(\lambda) = 0$. \\
$\implies$ According to the Fundamental Theorem of Algebra, there are k roots for $P$ in $\C$. \\
\Bem $P(\lambda)$ is the \textbf{characteristic polynomial} and the roots are called \textbf{eigenvalues}\\

\sep

\Theorem Let $\lambda_1, ..., \lambda_r$ be the pairwise distinct roots of $P(\lambda)$ with corresponding multiplicity $m_1, ..., m_r$.
Then the functions
$$x^{l}  e^{\lambda_j x} \quad 1 \leq j \leq r, \quad 0 \leq l < m_j$$
form a basis of the space of solutions of the homogeneous equation. \\
E.g. for k distinct roots we get:
$$ f(x) = z_1 e^{\lambda_1 x} + \cdots + z_k e^{\lambda_k x}, \text{ with }  z_1, ..., z_2 \in \C $$

\sep

\Remark If we are only interested in real solutions, the solutions based on complex roots, the basis can be transformed. For $\lambda=a + b i$:
$$ \text{span}(e^{\lambda x}, e^{\bar{\lambda}x}) =
   \text{span}(e^{a x} \cos{(b x)}, e^{a x} \sin{(b x)})$$
   
\subsection{Solving the inhomogenous eqn.}
\Procedure[Ansatz]
\begin{table}[H]
  \begin{center}
    \begin{tabular}{l|l}
      $b(x)$ & Ansatz $y_p(x)$ \\
      \hline
      
      $P_n(x)$ & $Q_n(x)$ \\
      $P_n(x)e^{\mu x}$ & $Q_n(x)e^{\mu x}$\\
      \hline
      
      $\begin{matrix}
      P_n(x) \sin(\nu x) \\ + 
      Q_n(x)\cos(\nu x)
      \end{matrix}$ 
      & $\begin{matrix}R_n(x)\sin(\nu x)\qquad\\ 
      + S_n(x)\cos(\nu x)\end{matrix}$ \\
      \hline
      
      $\begin{matrix}P_n(x)e^{\mu x}\sin(\nu x) \\
       + Q_n(x)e^{\mu x}\cos(\nu x)\end{matrix} $
      & $\begin{matrix}e^{\mu x}(R_n(x)\sin(\nu x) \\ + S_n(x)\cos(\nu x))\end{matrix}$
    \end{tabular}
  \end{center}
\end{table}
Insert $y_p(x)$ in the inhomogeneous eqn. and solve for the constants. $P_n(x), Q_n(x), \dots$ are polynomials of degree $n$. \\
\Remark If $y_p(x)$ is a root of $P(\lambda)$ of multiplicity $m$, multiply $y_p(x)$ by $x^m$.
\sep
\Procedure[Variation of constants] \\
This method can be derived from the matrix describing the problem. Assume $n=2$. \\
Try $y_p = z_{1}(x) f_{1}+z_{2}(x)f_{2}$ after solving the system:
$$
\left\{\begin{array}{l}
z_{1}^{\prime}(x) f_{1}+z_{2}^{\prime}(x) f_{2}=0 \\
z_{1}^{\prime}(x) f_{1}^{\prime}+z_{2}^{\prime}(x) f_{2}^{\prime}=b
\end{array}\right.
$$

\subsection{Separation of Variables}
\Recipe
One can try to separate the variables (e.g only $y$' on the left, only $x$'s on the right) in order to solve a non-linear first order ODE. We get:
$$\int \frac{\mathrm{d} y}{g(y)}=\int \mathrm{d} x b(x)$$
If for any $y_0$ it is $g(y_0)=0$, the constant function $y=y_0$ is a solution.